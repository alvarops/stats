%%%%%%%%%%%%%%%%%%%%%%%%%%%%%%%%%%%%%%%%%
% Scalable Computing Assignment
%
% This file is based on a template downloaded from:
% http://www.latextemplates.com
%
% Original author of the template:
% Ted Pavlic (http://www.tedpavlic.com)
%
%%%%%%%%%%%%%%%%%%%%%%%%%%%%%%%%%%%%%%%%%

%----------------------------------------------------------------------------------------
%	PACKAGES AND OTHER DOCUMENT CONFIGURATIONS
%----------------------------------------------------------------------------------------

\documentclass{article}

\usepackage{fancyhdr} % Required for custom headers
\usepackage{lastpage} % Required to determine the last page for the footer
\usepackage{extramarks} % Required for headers and footers
\usepackage[usenames,dvipsnames]{color} % Required for custom colors
\usepackage{graphicx} % Required to insert images
\usepackage{listings} % Required for insertion of code
\usepackage{courier} % Required for the courier font

% Margins
\topmargin=-0.45in
\evensidemargin=0in
\oddsidemargin=0in
\textwidth=6.5in
\textheight=9.0in
\headsep=0.25in

\linespread{1.1} % Line spacing

% Set up the header and footer
\pagestyle{fancy}
\lhead{\hmwkAuthorName} % Top left header
\chead{\hmwkClass\ (\hmwkClassInstructor\ \hmwkClassTime): \hmwkTitle} % Top center head
\rhead{\firstxmark} % Top right header
\lfoot{\lastxmark} % Bottom left footer
\cfoot{} % Bottom center footer
\rfoot{Page\ \thepage\ of\ \protect\pageref{LastPage}} % Bottom right footer
\renewcommand\headrulewidth{0.4pt} % Size of the header rule
\renewcommand\footrulewidth{0.4pt} % Size of the footer rule

\setlength\parindent{0pt} % Removes all indentation from paragraphs

%----------------------------------------------------------------------------------------
%	CODE INCLUSION CONFIGURATION
%----------------------------------------------------------------------------------------

\definecolor{MyDarkGreen}{rgb}{0.0,0.4,0.0} % This is the color used for comments
\lstloadlanguages{C} % Load C syntax for listings, for a list of other languages supported see: ftp://ftp.tex.ac.uk/tex-archive/macros/latex/contrib/listings/listings.pdf
\lstset{language=C, % Use C in this example
        frame=single, % Single frame around code
        basicstyle=\small\ttfamily, % Use small true type font
        keywordstyle=[1]\color{Blue}\bf, % C functions bold and blue
        keywordstyle=[2]\color{Purple}, % C function arguments purple
        keywordstyle=[3]\color{Blue}\underbar, % Custom functions underlined and blue
        identifierstyle=, % Nothing special about identifiers                                         
        commentstyle=\usefont{T1}{pcr}{m}{sl}\color{MyDarkGreen}\small, % Comments small dark green courier font
        stringstyle=\color{Purple}, % Strings are purple
        showstringspaces=false, % Don't put marks in string spaces
        tabsize=5, % 5 spaces per tab
        %
        % Put standard C functions not included in the default language here
        morekeywords={rand},
        %
        % Put C function parameters here
        morekeywords=[2]{on, off, interp},
        %
        % Put user defined functions here
        morekeywords=[3]{test},
       	%
        morecomment=[l][\color{Blue}]{...}, % Line continuation (...) like blue comment
        numbers=left, % Line numbers on left
        firstnumber=1, % Line numbers start with line 1
        numberstyle=\tiny\color{Blue}, % Line numbers are blue and small
        stepnumber=5 % Line numbers go in steps of 5
}

% Creates a new command to include a C script, the first parameter is the filename of the script (without .c), the second parameter is the caption
\newcommand{\cscript}[2]{
\begin{itemize}
\item[]\lstinputlisting[caption=#2,label=#1]{#1.c}
\item[]\lstinputlisting[caption=Corresponding header file,label=#1.h]{#1.h}
\end{itemize}
}

%----------------------------------------------------------------------------------------
%	DOCUMENT STRUCTURE COMMANDS
%	Skip this unless you know what you're doing
%----------------------------------------------------------------------------------------

% Header and footer for when a page split occurs within a problem environment
\newcommand{\enterProblemHeader}[1]{
\nobreak\extramarks{#1}{#1 continued on next page\ldots}\nobreak
\nobreak\extramarks{#1 (continued)}{#1 continued on next page\ldots}\nobreak
}

% Header and footer for when a page split occurs between problem environments
\newcommand{\exitProblemHeader}[1]{
\nobreak\extramarks{#1 (continued)}{#1 continued on next page\ldots}\nobreak
\nobreak\extramarks{#1}{}\nobreak
}

\setcounter{secnumdepth}{0} % Removes default section numbers
\newcounter{homeworkProblemCounter} % Creates a counter to keep track of the number of problems

\newcommand{\homeworkProblemName}{}
\newenvironment{homeworkProblem}[1][Question \arabic{homeworkProblemCounter}]{ % Makes a new environment called homeworkProblem which takes 1 argument (custom name) but the default is "Problem #"
\stepcounter{homeworkProblemCounter} % Increase counter for number of problems
\renewcommand{\homeworkProblemName}{#1} % Assign \homeworkProblemName the name of the problem
\section{\homeworkProblemName} % Make a section in the document with the custom problem count
\enterProblemHeader{\homeworkProblemName} % Header and footer within the environment
}{
\exitProblemHeader{\homeworkProblemName} % Header and footer after the environment
}

\newcommand{\problemAnswer}[1]{ % Defines the problem answer command with the content as the only argument
\noindent\framebox[\columnwidth][c]{\begin{minipage}{0.98\columnwidth}#1\end{minipage}} % Makes the box around the problem answer and puts the content inside
}

\newcommand{\homeworkSectionName}{}
\newenvironment{homeworkSection}[1]{ % New environment for sections within homework problems, takes 1 argument - the name of the section
\renewcommand{\homeworkSectionName}{#1} % Assign \homeworkSectionName to the name of the section from the environment argument
\subsection{\homeworkSectionName} % Make a subsection with the custom name of the subsection
\enterProblemHeader{\homeworkProblemName\ [\homeworkSectionName]} % Header and footer within the environment
}{
\enterProblemHeader{\homeworkProblemName} % Header and footer after the environment
}


%----------------------------------------------------------------------------------------
%	NAME AND CLASS SECTION
%----------------------------------------------------------------------------------------

\newcommand{\hmwkTitle}{CA\ \#3} % Assignment title
\newcommand{\hmwkDueDate}{Monday,\ May\ 6th,\ 2013} % Due date
\newcommand{\hmwkClass}{Scalable Computing} % Course/class
\newcommand{\hmwkClassTime}{} % Class/lecture time
\newcommand{\hmwkClassInstructor}{Dr. John Burns} % Teacher/lecturer
\newcommand{\hmwkAuthorName}{Alvaro Pereda} % Your name

%----------------------------------------------------------------------------------------
%	TITLE PAGE
%----------------------------------------------------------------------------------------

\title{
\vspace{2in}
\textmd{\textbf{\hmwkClass:\ \hmwkTitle}}\\
\normalsize\vspace{0.1in}\small{Due\ on\ \hmwkDueDate}\\
\vspace{0.1in}\large{\textit{\hmwkClassInstructor\ \hmwkClassTime}}
\vspace{3in}
}

\author{\textbf{\hmwkAuthorName}}
\date{} % Insert date here if you want it to appear below your name

%----------------------------------------------------------------------------------------

\begin{document}

\maketitle

%----------------------------------------------------------------------------------------
%	TABLE OF CONTENTS
%----------------------------------------------------------------------------------------

%\setcounter{tocdepth}{1} % Uncomment this line if you don't want subsections listed in the ToC

\newpage
\tableofcontents
\newpage

%----------------------------------------------------------------------------------------
%	PROBLEM 1
%----------------------------------------------------------------------------------------

% To have just one problem per page, simply put a \clearpage after each problem

\begin{homeworkProblem}

\begin{homeworkSection}{Part 1:}
We wish to find the maximum, minimum and mean value in an array of 500 random integers between 0 and 1000. Write a C/C++ program to achieve this. You are NOT permitted to use any libraries for
this - you must implement all aspects of the program yourself. If you use any libraries you will score 0 for this section. Use the timing methodology from the labs to demonstrate the wall-clock performance of this solution.\\
\problemAnswer{
Listing \ref{noscal/statistics-threads} shows the algorithm implementation in C. The problem has been divided in three functions:
\begin{itemize}
\item The \emph{main} function, that schedules the calls to the other functions, sets the timers up and displays the results.
\item The \emph{build} function, that with calls to the mathematical pseudo random numbers generatior \emph{rand()} initializes the integer array.
\item The \emph{stats} function, that collects the max value, min and average.
\end{itemize}
In order to collect the results, a struct has been implemented as can be seen in Listing \ref{noscal/statistics-threads.h}. The loop printing the results has been commented out.\\
The results are as follows: \\
Start\\
Avg = 47\\
\\
Max = 99\\
\\
Min = 0\\
Static: Elapsed: 0.000082 seconds\\
}
\cscript{noscal/statistics-threads}{C source file}
\end{homeworkSection}
\begin{homeworkSection}{Part 2:}
Next, discuss in no less than 500 words how each and every part of your program could be executed in parallel.\\
\problemAnswer{

}
\end{homeworkSection}
\begin{homeworkSection}{Part 3.}
Using pthreads, implement your analysis from part 2. to build a parallel solution. Demonstrate the results of this in your answer PDF to show the maximum, minimum and mean are correct. To do this, it is best to run the serial code followed by the parallel code. In this section you will be marked on correctness of the solution. You must make sure that workload is evenly distributed across threads and the threads must all execute separate and independent computations.\\
\problemAnswer{
Even though both Selection Sort and this algorithm have the same cost, Selection Sort performs better in writing count, making only 2 n writings in the average case. For example we can use the following array, where each line represents a cycle of the outer loop:\\
\emph{Selection Sort}\\
2 6 5 1 4 3\\
1 6 5 2 4 3\\
1 2 5 6 4 3\\
1 2 3 6 4 5\\
1 2 3 4 6 5\\
1 2 3 4 5 6\\
\emph{Algorithm Sort}\\
2 6 5 1 4 3\\
2 6 5 1 4 3\\
2 5 6 1 4 3\\
1 2 5 6 4 3\\
1 2 4 5 6 3\\
1 2 3 4 5 6}
\end{homeworkSection}
\end{homeworkProblem}

%----------------------------------------------------------------------------------------

\end{document}