%%%%%%%%%%%%%%%%%%%%%%%%%%%%%%%%%%%%%%%%%
% Scalable Computing Assignment
%
% This file is based on a template downloaded from:
% http://www.latextemplates.com
%
% Original author of the template:
% Ted Pavlic (http://www.tedpavlic.com)
%
%%%%%%%%%%%%%%%%%%%%%%%%%%%%%%%%%%%%%%%%%

%----------------------------------------------------------------------------------------
%	PACKAGES AND OTHER DOCUMENT CONFIGURATIONS
%----------------------------------------------------------------------------------------

\documentclass{article}

\usepackage{fancyhdr} % Required for custom headers
\usepackage{lastpage} % Required to determine the last page for the footer
\usepackage{extramarks} % Required for headers and footers
\usepackage[usenames,dvipsnames]{color} % Required for custom colors
\usepackage{graphicx} % Required to insert images
\usepackage{listings} % Required for insertion of code
\usepackage{courier} % Required for the courier font

% Margins
\topmargin=-0.45in
\evensidemargin=0in
\oddsidemargin=0in
\textwidth=6.5in
\textheight=9.0in
\headsep=0.25in

\linespread{1.1} % Line spacing

% Set up the header and footer
\pagestyle{fancy}
\lhead{\hmwkAuthorName} % Top left header
\chead{\hmwkClass\ (\hmwkClassInstructor\ \hmwkClassTime): \hmwkTitle} % Top center head
\rhead{\firstxmark} % Top right header
\lfoot{\lastxmark} % Bottom left footer
\cfoot{} % Bottom center footer
\rfoot{Page\ \thepage\ of\ \protect\pageref{LastPage}} % Bottom right footer
\renewcommand\headrulewidth{0.4pt} % Size of the header rule
\renewcommand\footrulewidth{0.4pt} % Size of the footer rule

\setlength\parindent{0pt} % Removes all indentation from paragraphs

%----------------------------------------------------------------------------------------
%	CODE INCLUSION CONFIGURATION
%----------------------------------------------------------------------------------------

\definecolor{MyDarkGreen}{rgb}{0.0,0.4,0.0} % This is the color used for comments
\lstloadlanguages{C} % Load C syntax for listings, for a list of other languages supported see: ftp://ftp.tex.ac.uk/tex-archive/macros/latex/contrib/listings/listings.pdf
\lstset{language=C, % Use C in this example
        frame=single, % Single frame around code
        basicstyle=\small\ttfamily, % Use small true type font
        keywordstyle=[1]\color{Blue}\bf, % C functions bold and blue
        keywordstyle=[2]\color{Purple}, % C function arguments purple
        keywordstyle=[3]\color{Blue}\underbar, % Custom functions underlined and blue
        identifierstyle=, % Nothing special about identifiers                                         
        commentstyle=\usefont{T1}{pcr}{m}{sl}\color{MyDarkGreen}\small, % Comments small dark green courier font
        stringstyle=\color{Purple}, % Strings are purple
        showstringspaces=false, % Don't put marks in string spaces
        tabsize=5, % 5 spaces per tab
        %
        % Put standard C functions not included in the default language here
        morekeywords={rand},
        %
        % Put C function parameters here
        morekeywords=[2]{on, off, interp},
        %
        % Put user defined functions here
        morekeywords=[3]{test},
       	%
        morecomment=[l][\color{Blue}]{...}, % Line continuation (...) like blue comment
        numbers=left, % Line numbers on left
        firstnumber=1, % Line numbers start with line 1
        numberstyle=\tiny\color{Blue}, % Line numbers are blue and small
        stepnumber=5 % Line numbers go in steps of 5
}

% Creates a new command to include a C script, the first parameter is the filename of the script (without .c), the second parameter is the caption
\newcommand{\cscript}[2]{
\begin{itemize}
\item[]\lstinputlisting[caption=#2,label=#1]{#1.c}
\end{itemize}
}

%----------------------------------------------------------------------------------------
%	DOCUMENT STRUCTURE COMMANDS
%	Skip this unless you know what you're doing
%----------------------------------------------------------------------------------------

% Header and footer for when a page split occurs within a problem environment
\newcommand{\enterProblemHeader}[1]{
\nobreak\extramarks{#1}{#1 continued on next page\ldots}\nobreak
\nobreak\extramarks{#1 (continued)}{#1 continued on next page\ldots}\nobreak
}

% Header and footer for when a page split occurs between problem environments
\newcommand{\exitProblemHeader}[1]{
\nobreak\extramarks{#1 (continued)}{#1 continued on next page\ldots}\nobreak
\nobreak\extramarks{#1}{}\nobreak
}

\setcounter{secnumdepth}{0} % Removes default section numbers
\newcounter{homeworkProblemCounter} % Creates a counter to keep track of the number of problems

\newcommand{\homeworkProblemName}{}
\newenvironment{homeworkProblem}[1][Question \arabic{homeworkProblemCounter}]{ % Makes a new environment called homeworkProblem which takes 1 argument (custom name) but the default is "Problem #"
\stepcounter{homeworkProblemCounter} % Increase counter for number of problems
\renewcommand{\homeworkProblemName}{#1} % Assign \homeworkProblemName the name of the problem
\section{\homeworkProblemName} % Make a section in the document with the custom problem count
\enterProblemHeader{\homeworkProblemName} % Header and footer within the environment
}{
\exitProblemHeader{\homeworkProblemName} % Header and footer after the environment
}

\newcommand{\problemAnswer}[1]{ % Defines the problem answer command with the content as the only argument
\noindent\framebox[\columnwidth][c]{\begin{minipage}{0.98\columnwidth}#1\end{minipage}} % Makes the box around the problem answer and puts the content inside
}

\newcommand{\homeworkSectionName}{}
\newenvironment{homeworkSection}[1]{ % New environment for sections within homework problems, takes 1 argument - the name of the section
\renewcommand{\homeworkSectionName}{#1} % Assign \homeworkSectionName to the name of the section from the environment argument
\subsection{\homeworkSectionName} % Make a subsection with the custom name of the subsection
\enterProblemHeader{\homeworkProblemName\ [\homeworkSectionName]} % Header and footer within the environment
}{
\enterProblemHeader{\homeworkProblemName} % Header and footer after the environment
}


%----------------------------------------------------------------------------------------
%	NAME AND CLASS SECTION
%----------------------------------------------------------------------------------------

\newcommand{\hmwkTitle}{CA\ \#1} % Assignment title
\newcommand{\hmwkDueDate}{Wednesday,\ March\ 20th,\ 2013} % Due date
\newcommand{\hmwkClass}{Scalable Computing} % Course/class
\newcommand{\hmwkClassTime}{} % Class/lecture time
\newcommand{\hmwkClassInstructor}{Dr. John Burns} % Teacher/lecturer
\newcommand{\hmwkAuthorName}{Alvaro Pereda} % Your name

%----------------------------------------------------------------------------------------
%	TITLE PAGE
%----------------------------------------------------------------------------------------

\title{
\vspace{2in}
\textmd{\textbf{\hmwkClass:\ \hmwkTitle}}\\
\normalsize\vspace{0.1in}\small{Due\ on\ \hmwkDueDate}\\
\vspace{0.1in}\large{\textit{\hmwkClassInstructor\ \hmwkClassTime}}
\vspace{3in}
}

\author{\textbf{\hmwkAuthorName}}
\date{} % Insert date here if you want it to appear below your name

%----------------------------------------------------------------------------------------

\begin{document}

\maketitle

%----------------------------------------------------------------------------------------
%	TABLE OF CONTENTS
%----------------------------------------------------------------------------------------

%\setcounter{tocdepth}{1} % Uncomment this line if you don't want subsections listed in the ToC

\newpage
\tableofcontents
\newpage

%----------------------------------------------------------------------------------------
%	PROBLEM 1
%----------------------------------------------------------------------------------------

% To have just one problem per page, simply put a \clearpage after each problem

\begin{homeworkProblem}

\begin{homeworkSection}{Part 1:}
Derive the best, worst AND average case O estimates for this sorting function.\\
\problemAnswer{
In this case, the main loop will give the Algorithm cost. Outer loop depends on n, and inner loop counts from 0 to the current value of the outer loop index, so indirectly also depends on n. \\
This is a clear case of a Eqn.~\ref{eqn:i}:
\begin{equation}
\label{eqn:i}
\sum_{i=1}^n n
\end{equation}

Equation~\ref{eqn:i} can be represented as Equation~\ref{eqn:j}:
\begin{equation}
\label{eqn:j}
(n + 1)*n/2
\end{equation}

Which can be simplified to Equation~\ref{eqn:k}:
\begin{equation}
\label{eqn:k}
n^2
\end{equation}

So in all cases the cost will be:
\begin{equation}
\label{eqn:o}
O(n^2)
\end{equation}
}
\end{homeworkSection}
\begin{homeworkSection}{Part 2:}
Is this algorithm sensitive to input data permutation (or ordering)? Explain your answer.\\
\problemAnswer{
The cost will not change, as it will perform all comparisons no matter the status of the array. There is not a flag that indicates there has not been a swap at some point, so the algorithm would stop, and perform better in the best cases.
}
\end{homeworkSection}
\begin{homeworkSection}{Part 3.}
Discuss how this algorithm compares to the sorting methodology of the Selection sort we saw in the lecture notes. Give an example.\\
\problemAnswer{
Even though both Selection Sort and this algorithm have the same cost, Selection Sort performs better in writing count, making only 2 n writings in the average case. For example we can use the following array, where each line represents a cycle of the outer loop:\\
\emph{Selection Sort}\\
2 6 5 1 4 3\\
1 6 5 2 4 3\\
1 2 5 6 4 3\\
1 2 3 6 4 5\\
1 2 3 4 6 5\\
1 2 3 4 5 6\\
\emph{Algorithm Sort}\\
2 6 5 1 4 3\\
2 6 5 1 4 3\\
2 5 6 1 4 3\\
1 2 5 6 4 3\\
1 2 4 5 6 3\\
1 2 3 4 5 6}
\end{homeworkSection}
\end{homeworkProblem}
%----------------------------------------------------------------------------------------
%	PROBLEM 2
%----------------------------------------------------------------------------------------
\begin{homeworkProblem}
Implement the above algorithm in C or C++ making sure your program outputs the unsorted data followed by sorted output.
You may modify the algorithm only to the extent of adding relevant supporting functions, some additional lines etc. Obviously you will need to
include your own implementation of FillRandomData and swap. Please do not extensively alter the algorithm critical operations.
You must use the gcc or g++ compiler no other compiler technology will be accepted. If you used any special command line compile flags (such as stdc99= or -lm) you need to include this at the top of the source code.

\problemAnswer{
Attached to this pdf file, can be found the source code of sort.c and its header file sort.h. The methods added to the algorithm are \textit{build} and \textit{swap}, whose cost is \textit{O(n)} and \textit{O(1)} respectively.
Listing \ref{sort/sort} shows the algorithm implementation in C.
}
\cscript{sort/sort}{Sample C Script With Highlighting}


\end{homeworkProblem}

%----------------------------------------------------------------------------------------

\end{document}